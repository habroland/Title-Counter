\documentclass{article}
\usepackage[utf8]{inputenc}
\usepackage[english]{babel}
\usepackage[margin=1in]{geometry}

\author{RH}

\newcounter{stepnum}
\newenvironment{step}[1][]{\refstepcounter{stepnum}\par\medskip
   \noindent\textbf{Step~\thestepnum #1:} \rmfamily}{\medskip} % Alternatively you can use {Step~\thestepnum~of \steptotal #1:}
   
\makeatletter
\AtEndDocument{
  \immediate\write\@auxout{% \immediate seems to be unnecessary
    \global\noexpand\@namedef{steptotal}{\number\c@stepnum}% Also can use \thestepnum instead of \number\c@stepnum
  }%
}
\title{How to make your own quantum computer in \@nameuse{steptotal} hard steps}
\makeatother

\begin{document}

\maketitle

\section{Introduction}

This \LaTeX \, preamble introduces an environment called ``step'', whose number of instances will be reflected in the title.

\section{Proceduce}

Steps to make your very own quantum computer

\begin{step} \label{step:1}
Purchase the qubits % This text is inside the environment

Step number \ref{step:1} is to get qubits from your local store. % Test reference to label
\end{step}
 
Remember that everything is quantum! % This text is outside the environment
 
\begin{step}
Assemble the qubits

Step \thestepnum \, tells you to put everything together, or else go to your nearest quantum mechanic and let them assemble the qubits. % Test counter
\end{step}
 
\begin{step}[ of \steptotal] % Test argument of environment and test steptotal
?????
\end{step}

\begin{step}
Profit!
\end{step}

\section{Conclusion}

Enjoy your quantum gaming, and your new step environment.

\section{Epilogue}

Originally, I intended to name this file ``How to make your own quantum computer in \steptotal \, easy steps'', but my wife pointed out that these steps are not easy. Hence, I fixed the title.

\end{document}